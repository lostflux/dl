\documentclass[11pt, reqno]{amsart}

\input{~/latex-common/macros.tex}
\usepackage[backend=bibtex,style=science]{biblatex}
% \bibliography{main.bib}
\pgfplotsset{compat=1.18}

\pagestyle{fancy}                         % fancy (allow headers, footers)
\fancyhf{}                                % clear all header/footer settings.
\cfoot{\thepage}                          % set page-numbers in footer.
% \lhead{\textit{\textbf{ Amittai, S}}}   % set name in header, left.
% \rhead{\textsc{Math 71: Algebra}}       % set class name in header, right.
\renewcommand{\headrulewidth}{0pt}
\renewcommand{\footrulewidth}{0pt}


\renewcommand{\theenumi}{\alph{enumi}}

\begin{document}

\newdate{due-date}{27}{04}{2023}

\title{CS-89.31: Deep Learning Generalization and Robustness\\ Amittai Siavava \\ \displaydate{due-date}}
\author{Amittai Siavava}
% \date{\today}


\setlength{\headheight}{13.0pt}
\setlength{\footskip}{15.0pt}

\maketitle

\bigskip

\def \cram { \textsc{cram} }
\def \dom { \textsc{domineering} }
\def \sub { \textsc{subtraction} }
\def \weighted { \textsc{weighted odds and evens}}
\def \nim { \textsc{nim} }
\def \P { \mathbf{P}}
\def \N { \mathbf{N}}

\newpage
\begin{problem}[1]
  Model results and discussion.

  \step
  For even $n$, the next player $\N$ can win as follows:
  \begin{enumroman}
    \item Let the squares be numbered $1, 2, \ldots, n$ from left to right.
    \item $\N$ marks the two contiguous squares $n/2, n/2 + 1$.
    \item This leaves two identical positions, each of dimensions
      $1 \times \parens{\frac{n}{2} - 1}$.
      Since $\P$ moves first in this new situation, $\N$ can win by strategy-stealing.
  \end{enumroman}
  
  \step
  For odd $n$, the best strategy for the next player is to leave a position with an odd
  dimension for the other player, since leaving a position with even dimensions
  loses by the strategy above.
  The outcome therefore depends on how many such moves can be made before running out
  --- that is,
  \[ \mathbf{outcome}(G) = \begin{cases}
      \N & \text{if \quad $n \pmod 2 \equiv 1$} \\
      \P & \text{if \quad $n \pmod 2 \equiv 0$}
  \end{cases} \]
  Particularly, $(1 \times 7) \in \N$ and $(1 \times 9) \in \P$.

  \step
  Therefore, the values of the games are:
  \begin{table}[h]
    \centering
    \begin{tabular}{|l|r|r|}
      Dimensions & Class & Value \\
      \bottomrule
      $(1 \times 7)$ & $\N$ & $1$ \\
      $(1 \times 8)$ & $\N$ & $1$ \\
      $(1 \times 9)$ & $\P$ & $0$ \\
      $(1 \times 10)$ & $\N$ & $1$ \\
      \toprule
    \end{tabular}
    \caption{Values of $1 \times n$ \cram for $n = 7, 8, 9, 10$}
  \end{table}
\end{problem}
\end{document}
